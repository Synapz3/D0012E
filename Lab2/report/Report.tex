\documentclass[paper=a4, fontsize=11pt]{scrartcl}

\usepackage[T1]{fontenc}
\usepackage{fourier}
\usepackage[utf8]{inputenc}

\usepackage[english]{babel}															% English language/hyphenation
\usepackage[protrusion=true,expansion=true]{microtype}	
\usepackage{amsmath,amsfonts,amsthm} % Math packages
\usepackage[pdftex]{graphicx}	
\usepackage{url}
\usepackage{abstract}
\usepackage{tabularx}
\usepackage{float}

%%% Custom sectioning
\usepackage{sectsty}
\allsectionsfont{\centering \normalfont\scshape}

%%% Custom headers/footers (fancyhdr package)
\usepackage{fancyhdr}
\pagestyle{fancyplain}
\fancyhead{}											% No page header
\fancyfoot[L]{}											% Empty 
\fancyfoot[C]{}											% Empty
\fancyfoot[R]{\thepage}									% Pagenumbering
\renewcommand{\headrulewidth}{0pt}			% Remove header underlines
\setlength{\headheight}{13.6pt}

%%% Equation and float numbering
\numberwithin{equation}{section}		% Equationnumbering: section.eq#
\numberwithin{figure}{section}			% Figurenumbering: section.fig#
\numberwithin{table}{section}				% Tablenumbering: section.tab#

%%% Maketitle metadata
\newcommand{\horrule}[1]{\rule{\linewidth}{#1}} 	% Horizontal rule

\title{
		%\vspace{-1in} 	
		\usefont{OT1}{bch}{b}{n}
		\normalfont \normalsize \textsc{Luleå University of Technology} \\ [25pt]
		\horrule{0.5pt} \\[0.4cm]
		\huge Algorithms and Data Structures - Laboration 2 \\
		\horrule{2pt} \\[0.5cm]
}
\author{Course: D0012E \\ \\ 
		\normalfont 								\normalsize
        Marcus Lund (amuulo-4) \\\normalfont\normalsize Edvin Åkerfeldt (edvker-4)\\\normalfont\normalsize Samuel Karlsson (samkar-4)\\[-3pt]		\normalsize
        \today
}
\date{}

%%% Begin document
\begin{document}
\maketitle
\centerline{First submission.}
\begin{figure}[h!]
  \centering
    \includegraphics[width=1\textwidth]{algorithm}
\end{figure}
\newpage
\tableofcontents
\newpage

\section{Introduction}
The Lab is to implement two variations of linear probing. Linear probing is a way to store data by filling a array, slot by slot. It can be imagined as a parkinglot where you arrive with a car and start to search for a empty slot from any random position. You continue to search for a empty slot till you find one or discover the parking lot is full.

The first variant in the lab is a search in both directions from the initial spot. In the second variant you only search $c$ (ware $c$ is a content of users chose) steps in both directions, in comparison to the the first variant where you search the entire array (linear). If no slot is available if an interval, the array is re-hashed and we try again with the new array.

\subsection{Language}
This lab we choose tried to implement the algorithms in C++, but discovered that our tests were wrong and shwitched to Java since we have more experience with Java compared to C++. In the previous lab we used Python and during the lab we discovered that Python is really bad optimised language. Therefore we decided to switch language, any other than Python. Another contributing factor to switch language between labs is to get a wider knowledge of diffrent languages. Also to reftesh old skills.

\subsection{Implementation}
The implementation of the two variants is a bit tricky because there is a lot of small details to keep track of. For example, a variable that count the number of an element in the array or a variable for the array size. Both are easy to kip track of but it needs to be done.\\~\\
Our first implementation takes an element and hashes it. The program then tries to place the element in an array at the position of the hash value. If the spot is not empty we look for the closest spot to that location.If we find an enmpty slot, we place the element at that slot. Otherwise the array is full.\\~\\
The secound implementation is simlair to the first one. It hashes the element and tries to place the element in an array at the hashed value. Then if the slot is taken, we check up and down to hopefully find the closest slot that´s empty. Once we know the distance between the original hashed value and the free slot, we check if the distance is grater than a predetermined constatnt. If there is a free space, than place the element at that psoition. If there is no available slot within the range of $x$ and $x+c$, then find an element in the interval $x$ and $x+c$ with the condition that $y$ (the possible position) has an empty slot in the intervall $y$ and $y+c$. If sutch an element exists, move the element at $y$ to the empty slot in the intervall $y$ and $y+c$. Then place the new element at the $y$, witch is now empty. Otherwise, rehash the element and try again.



\section{Test procedure}
The tests were conducted using different hashtable sizes: $10$,$100$,$1000$,$10000$,$100000$ and $1000000$. $10000$ keys were randomly generated between $0$ and $1 000 000 000$, we chose to stop at $10000$ because we thought it wouldn't contribute anything to the end result to go higher. The keys were then hashed to find a home address to insert it into. Also the testing function stops trying to inserting new values if the hashingtable is full.
\subsection{Implementation 1}
When it comes to implementation 1 no other values then the hashtable size was to impact the end result, so therefore tests were only conducted with the different hashtable sizes.
\subsection{Implementation 2}
Although when it comes to implementation 2 a c value was thrown into the mix and therefore tests were conducted with that aswell. The c value was changed with a constant hashtable size.
 
\section{Result}
\subsection{Implementation 1}
\begin{table}[]
\centering
\caption{Result from Implementation 1 }
\label{impl1_table}
\begin{tabular}{lllll}
Probes & Hashtable size & max\_collision chain & insertion\_collisions & runtime \\
10     & 10             & 5                   & 6                     & 0 ms    \\
100    & 100            & 92                  & 42                    & 0 ms    \\
1000   & 1000           & 797                 & 530                   & 15 ms   \\
10 000  & 10 000         & 8744                & 5596                  & 0 ms    \\
10 000  & 100 000        & 6                   & 2417                  & 16 ms   \\
10 000  & 1 000 000      & 3                   & 2227                  & 16 ms  
\end{tabular}
\end{table}
Two things to take away from this result is firstly that runtime can be seen to be unconsistent and might not be completely reliable, secondly is that we can see a big drop in $max_collision_chain$ when the keys no longer fills the hashtable, this is because we no longer get so much overlap.
\subsection{Implementation 2}
First test were preformed with a constant c of $500$.
\begin{table}[]
\centering
\caption{My caption}
\label{my-label}
\begin{tabular}{lllll}
Probes & Hashtable size & rehashes made & insertions failed & runtime \\
10     & 10             & 0             & 0                 & 15 ms   \\
100    & 100            & 0             & 0                 & 0 ms    \\
1000   & 1000           & 41            & 41                & 79 ms   \\
10000  & 10 000         & 83            & 83                & 218 ms  \\
10000  & 100 000        & 0             & 0                 & 0 ms    \\
10000  & 1 000 000      & 0             & 0                 & 0 ms  
\end{tabular}
\end{table}
 
Second test is preformed with varying c and constant hashtable size and probes of $10000$.
 
\begin{table}[]
\centering
\caption{Result from implementation 2}
\label{impl2_table}
\begin{tabular}{llll}
C   & insertions\_faild & rehashes made & runtime \\
2   & 2571              & 2571          & 2407 ms \\
4   & 1514              & 1514          & 1813 ms \\
8   & 917               & 917           & 1281 ms \\
16  & 599               & 599           & 1000 ms \\
32  & 415               & 415           & 782 ms  \\
64  & 243               & 243           & 531 ms  \\
128 & 183               & 183           & 453 ms
\end{tabular}
\end{table}
 
In the last result table we can see that the c has big impact on both runningtime and insertion fails.



\section{Discussion}
We had some problems with the code witch gave as a ridicules amount of collisions. We discovered that it was because of ours hashing method that gave a really bad variation of integers. Thiess result in a bad load factor in the second variation. When we fixed the hashing method the number of collisions went down to resemble values.\\

When the hash table size is big and the elements to insert small is the number of collisions really small. this was expected, to image it on a big parking with fuel cars is it easy to find an empty slot but on a small parking with many cars is it hard to find an empty slot to park in.\\

While the hash table has a low load factor is liner probing a good way to store data. When the load factor is hay is the many collisions and many insertion fails (second variation). The extra works collision cause is really bad for the runtime. The insertion fails are wary bad for the run time especially if you don’t give up after one rehashing. We did only one rehashing before giving up on that element. Our rehashing has a low rat of fixing the insertion fail, if the hash table size increases wen a rehashing happens is the rehashing more likely to fix the insertion fail.



%%% End document
\end{document}
